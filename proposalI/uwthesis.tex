%  ========================================================================
%  Copyright (c) 1985 The University of Washington
%
%  Licensed under the Apache License, Version 2.0 (the "License");
%  you may not use this file except in compliance with the License.
%  You may obtain a copy of the License at
%
%      http://www.apache.org/licenses/LICENSE-2.0
%
%  Unless required by applicable law or agreed to in writing, software
%  distributed under the License is distributed on an "AS IS" BASIS,
%  WITHOUT WARRANTIES OR CONDITIONS OF ANY KIND, either express or implied.
%  See the License for the specific language governing permissions and
%  limitations under the License.
%  ========================================================================
%
\documentclass [11pt, proquest] {uwthesis}[2020/08/20]
\setcounter{tocdepth}{1} 
\usepackage{alltt}  % 
\newenvironment{demo}
  {\begin{alltt}\leftskip3em
     \def\\{\ttfamily\char`\\}%
     \def\{{\ttfamily\char`\{}%
     \def\}{\ttfamily\char`\}}}
  {\end{alltt}}
\let\mffont=\sf
\begin{document}
\prelimpages

\Title{Explorations In Entropy Based Training Methods For Language Representation}
\Author{Daniel Campos}
\Year{2020}
\Program{Department of Linguistics}
\Degree{Master of Science}
\Chair{Shane Steinert-Threlkeld}{Assistant Professor}{Department of Linguistics}
\Signature{Chenyan Xiong}
\Signature{Yan Song}

%\copyrightpage

\titlepage  

\setcounter{page}{-1}
%\abstract{blah}
 

%\tableofcontents
%\listoffigures
%\listoftables 
%\chapter*{Glossary} 
%\addcontentsline{toc}{chapter}{Glossary}
%\thispagestyle{plain}
%\begin{glossary}
%\item[LM] a language model
%\end{glossary}
 
%\acknowledgments{The author wishes to express sincere appreciation to University of Washington, the City of Seattle, and coffee.}

%\dedication{\begin{center}To my wife, Zoe\end{center}}

\textpages
\chapter {Introduction}
Shortly after the computer was introduced the research community was already imagining high levels of artificial intelligence. Early systems like ELIZA \cite{Weizenbaum1966ELIZAA} were designed to interact with humanity through verbal communication. In the decades that have elapsed we have grown to understand fundamental concepts about the structure of language and how best to model it. When word representations like Word2Vec \cite{Mikolov2013EfficientEO} and Glove \cite{Pennington2014GloveGV} were introduced, the Natural Language Processing(NLP) world was ushered into a new era. By providing general and task agnostic vector based representation for words, researchers began to push the boundaries on what language based systems could do. Using these word representations systems could now take advantage of the closeness of words and apply these representation to sub tasks in NLP State of the Art performance on common tasks like parsing and part of speech tagging TODO:CITE improved greatly and seemed to reach a new ceiling in language representation quality. \\ While these systems drove incredible improvement and research in the field, since representations focused on word level vector values, systems were unable to represent word as it different in context. The word fly was represented by the same vector independent if the context was dealing with an insect, an action, or a description of personal style. More recently, large scale language models like Elmo \cite{Peters2018DeepCW}, BERT, and GPT have provided methods of representing language in a context dependent way. These methods proved so effective that 2019 was by many considered the year long standing benchmarks were beaten. Inspired by the success of these initial language models, researchers started to tune and tweak the input data, the training methods, and training lengths producing a consistent string of new SOTAs. As the model experimentation continued it became clear that huge gains could be had by altering how the systems learned and what they learned. Efforts in model size alteration produced systems like MEGATRON, T5, and RoBERTA, while task reformulations like StructBERT. Each method produced a wildly different representation method which could be optimized to their desired task. Despite their differences all models proved difficult to train and required massive compute and long training times.   \\
At the same time NLP researchers were exploring neural methods for text representations the computer vision and deep learning community was exploring the concept of Generative Adversarial Networks(GAN) \cite{}. GANs are a deep learned method based on game theory where there is a Generator and a Discriminator. In training of a GAN a generator is optimized around producing samples within the distribution of the training data while the discriminator optimizes around classifying samples as within the dataset or not. The method is incredibly simple and has opened the way for machined learned systems to create. While the concept of these networks is simple, as the target output and dataset size grows so does the difficulty in training. There have been many successful methods in optimizing training of GANs \cite{}, \cite{}, \cite{}, with one of the most relevant being the progressive training of both the Discriminator and the Generator \cite{}. Unlike previous approaches and most traditional machine learned methods, Progressively trains models by increasing the size of the target output as training progresses. Initially, the Generator is generator is producing 2x2 pixel images. Once it and the Discriminator converge, the target output size is increased. This progressive training continues until the Generator is producing 4096 x 4096 images. By training in an increasingly entropic way, the Generator is able to continually learn an increasingly complex goal. \\
Much like large, realistic images, language can prove difficult to model. As a result, the goal of this paper is to explore the effect of progressive training on a variety of language representations.
\section {Goals}
-Explore training of language representations like Word2Vec, Glove, Sent2Vec, BERT, ELMO, ETC when trained on the same corpus and evaluated on a wide variety of benchmarks.
-Explore methods of progressive learning focused on selecting an increasingly complex textual sample.
-Evaluate linguistic properties of what each system can learn given various techniques of sample selection.
-At its core this method is seeking to explore and challenge the random sampling which is at the core of much of modern day deep learning and machine learning. 
\section {Challenges}
-What dataset to use?
-Time/Cost to train each of these networks
-Robustness to different inputs
-Reproducibility
-Adversarial attacks.
\chapter{Related Works}
* means central paper
\section{List of Papers}
\subsection{Transformers}
Universal Transformers* \\
Attention is All You Need* \\
Levenshtein Transformers \\
Transformers XL \\ 
\subsection{GANs}
Progressive Growing of GANs for Improved Quality, Stability, and Variation *
\subsection{Learning Theory}
Information BottleNeck Method*
ELECTRA: Pre-Traning Text Encoders as Discriminators Rather than Generators*\\
LSTMs Exploit Linguistic Attributes of Data\\
Uniform convergence may be unable to explain generalization in deep learning \\
Smaller Models, Better Generalization \\ 
Stabilizing the Lotterry Ticket Hypothesis \\
One ticket to win them all:generalizing lottery ticket initializations across datasets and optimizers. \\ 
Network Pruning via Transformable Architecture Search

\subsection{Language Representations}
StarSpace:Embed All The Things!\\
Glove \\
Word2Vec \\
Multimodal Word Distributions\\
ELMO*\\
BERT* \\
GPT\\
GPT-2*\\
RoBERTA \\
ALBERT*\\
DISTILLBERT*\\
Retrofitting Word Vectors to Semantic Lexicons \\
What the Vec? Towards Problematically grounded Embeddings \\
Specializing Word Embeddings (for Parsing) by information Bottleneck *\\
Knowledge Enhaced Contextual Word Represenations
\subsection{Evaluation}
Language Models as Knowledge Bases\\
oLMpics - On what Language Model Pre-Training Captures* \\
Open Sesame: Getting Inside BERT's Linguistic Knowledge. \\
Visualizing and Measuring the Geometry of BERT. \\
What Kind of Language Is Hard To Language Model \\
What Do Recurrent Neural Network Grammars Learn About Syntax? \\
Capacity, Bandwith, and Compositionality in Emergent Language Learning. * \\
Probing What Different NLP Tasks Teach Machines About Function Word Comprehension.
\section{Major Approaches}
Summary of major approaches in the papers 1-4 pages
\chapter{Research Approach}
1 page
Open web text dataset from \cite{Gokaslan2019OpenWeb}
\chapter{Evaluation}
\chapter{Conclusion}
\section{Summary of Contribution}
\section{Future Work}
\printendnotes
\nocite{*}
\bibliographystyle{plain}
\bibliography{uwthesis}
\appendix
\raggedbottom\sloppy
\chapter{Software}
blah
\vita{Daniel Campos is a Program Manager At Microsoft where he works on metrics and applied research on the Bing Search Engine. He welcomes your comments to {\tt spacemanidol@gmail.com}.}
\end{document}